\documentclass[12pt]{beamer}
\usepackage[utf8]{inputenc}
\usepackage[spanish]{babel}
\usepackage{mathtools}
\usepackage{relsize}
\usepackage{amsmath}

\usetheme{Copenhagen}
\setbeamertemplate{bibliography item}{\insertbiblabel}

\newcommand{\bra}[1]{\langle #1 |}
\newcommand{\ket}[1]{| #1 \rangle}
\newcommand{\braket}[2]{\langle #1 |#2\rangle}

\title{Clonación cuántica}
\author{Eric Brandwein}
\institute{FCEyN - UBA}
\date{15 de Diciembre de 2020}

\begin{document}
\frame{\titlepage}

\section{¿Por qué lo queremos?}
\begin{frame}
    \frametitle{¿Por qué lo queremos?}
    \begin{itemize}
        \item \textbf{Corrección de errores}, como lo hacemos en una computadora clásica.
        \item \textbf{Tomografía}, usando un solo estado en vez de
        necesitando muchas copias.
    \end{itemize}
\end{frame}

\section{¿Se puede hacer?}
\begin{frame}
    \frametitle{¿Se puede hacer?}
    % Supongamos que creamos una máquina que clona estados arbitrarios.
    % Ponele que esta máquina toma un estado cuántico en una entrada,
    % y por otra entrada toma basura.
    \[\mathlarger{\mathlarger{\mathlarger{
         \ket{\psi s} \overset{U}{\longrightarrow} U\ket{\psi s} = \ket{\psi\psi}
    }}}\]
    % Entonces tenemos que


\end{frame}

\begin{frame}
    \frametitle{¿Se puede hacer?}
    Supongamos que tenemos dos estados puros, \( \ket{\psi} \) y
    \( \ket{\varphi} \). Entonces:
    \begin{align*}
        U\ket{\psi s} = \ket{\psi\psi}\\
        U\ket{\varphi s} = \ket{\varphi\varphi}
    \end{align*}
    \pause
    Si tomamos el producto interno de estas dos ecuaciones, nos queda:
    \[
    \begin{aligned}
        \action<+->{
            U\ket{\psi s} \bullet U\ket{\varphi s} &=
            \ket{\psi\psi} \bullet \ket{\varphi\varphi} \\
        }
        \action<+->{
            \ket{\psi s} \bullet \ket{\varphi s} &=
            \braket{\psi}{\varphi}\braket{\psi}{\varphi} \\
        } % U preserva el producto interno por ser unitaria
        \action<+->{
            \braket{\psi}{\varphi} &=
            (\braket{\psi}{\varphi}) ^ 2
        }
    \end{aligned}
    \]
\end{frame}

\begin{frame}
    \frametitle{¿Se puede hacer?}
    \[\mathlarger{
        \braket{\psi}{\varphi} =
        (\braket{\psi}{\varphi}) ^ 2
    }\]
    \begin{itemize}
        \item ¿Cuándo puede pasar esto?
            Sólo si \( \braket{\psi}{\varphi} = 1 \) o
            \( \braket{\psi}{\varphi} = 0 \). \pause
        \item O sea, o \( \ket{\psi} = \ket{\varphi} \),
            o \( \ket{\psi} \) y \( \ket{\varphi} \) son ortogonales. \pause
        \item Es decir, nuestra máquina
            \textbf{sólo puede clonar dos estados ortogonales.}
            % Y esto quiere decir que no puede clonar cualquier estado
            % en general.
    \end{itemize}
\end{frame}

\begin{frame}
    \frametitle{¿Posta? ¿No podríamos hacerlo si...}
    ...usáramos transformaciones no unitarias? \pause \textbf{No.} \cite{demo-lineal}

    Hay otra demostración que no usa el hecho de que el operador es unitario, sino solo que es una transformación lineal.

\end{frame}
\begin{frame}
    \frametitle{¿Posta? ¿No podríamos hacerlo si...}
    ...clonásemos estados mixtos? \pause \textbf{Tampoco.} \cite{broadcast}

    Hay que tener cuidado cuando se define siquiera qué es clonar un estado
    mixto, pero igual no se puede.
\end{frame}

\begin{frame}
    \frametitle{¿Posta? ¿No podríamos hacerlo si...}
    ...permitiésemos que las copias no fuesen perfectas? \pause \textbf{ ¡Sí!} \cite{copying}
    Buzek y Hillery propusieron una máquina que copia un qubit con una fidelidad de 5/6 cuando se compara un solo qubit.
\end{frame}

\section{Referencias}
\begin{frame}
    \frametitle{Referencias}
    \begin{thebibliography}{9}
        \bibitem{demo-lineal}
        W. K. Wootters and W. H. Zurek. A single quantum cannot be cloned.
        Nature, 299:802–803, 1982

        \bibitem{broadcast}
        H.Barnum, C.M.Caves, C.A.Fuchs, R. Jozsa, and B. Schumacher. Noncommuting mixed states cannot be broadcast. Phys. Rev. Lett.,  76(15):2828–2821, 1996. arXive e-print quantph/9511010.

        \bibitem{copying}
        Bužek, V. and Hillery, M. (1996-09-01). Quantum copying: Beyond the no-cloning theorem. Physical Review A. 54 (3): 1844–1852. arXiv:quant-ph/9607018
    \end{thebibliography}

\end{frame}

\section{}
\begin{frame}
    \frametitle{¡Gracias!}
    \center ¿Preguntas?
\end{frame}



\end{document}